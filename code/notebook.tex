
% Default to the notebook output style

    


% Inherit from the specified cell style.




    
\documentclass[11pt]{article}

    
    
    \usepackage[T1]{fontenc}
    % Nicer default font (+ math font) than Computer Modern for most use cases
    \usepackage{mathpazo}

    % Basic figure setup, for now with no caption control since it's done
    % automatically by Pandoc (which extracts ![](path) syntax from Markdown).
    \usepackage{graphicx}
    % We will generate all images so they have a width \maxwidth. This means
    % that they will get their normal width if they fit onto the page, but
    % are scaled down if they would overflow the margins.
    \makeatletter
    \def\maxwidth{\ifdim\Gin@nat@width>\linewidth\linewidth
    \else\Gin@nat@width\fi}
    \makeatother
    \let\Oldincludegraphics\includegraphics
    % Set max figure width to be 80% of text width, for now hardcoded.
    \renewcommand{\includegraphics}[1]{\Oldincludegraphics[width=.8\maxwidth]{#1}}
    % Ensure that by default, figures have no caption (until we provide a
    % proper Figure object with a Caption API and a way to capture that
    % in the conversion process - todo).
    \usepackage{caption}
    \DeclareCaptionLabelFormat{nolabel}{}
    \captionsetup{labelformat=nolabel}

    \usepackage{adjustbox} % Used to constrain images to a maximum size 
    \usepackage{xcolor} % Allow colors to be defined
    \usepackage{enumerate} % Needed for markdown enumerations to work
    \usepackage{geometry} % Used to adjust the document margins
    \usepackage{amsmath} % Equations
    \usepackage{amssymb} % Equations
    \usepackage{textcomp} % defines textquotesingle
    % Hack from http://tex.stackexchange.com/a/47451/13684:
    \AtBeginDocument{%
        \def\PYZsq{\textquotesingle}% Upright quotes in Pygmentized code
    }
    \usepackage{upquote} % Upright quotes for verbatim code
    \usepackage{eurosym} % defines \euro
    \usepackage[mathletters]{ucs} % Extended unicode (utf-8) support
    \usepackage[utf8x]{inputenc} % Allow utf-8 characters in the tex document
    \usepackage{fancyvrb} % verbatim replacement that allows latex
    \usepackage{grffile} % extends the file name processing of package graphics 
                         % to support a larger range 
    % The hyperref package gives us a pdf with properly built
    % internal navigation ('pdf bookmarks' for the table of contents,
    % internal cross-reference links, web links for URLs, etc.)
    \usepackage{hyperref}
    \usepackage{longtable} % longtable support required by pandoc >1.10
    \usepackage{booktabs}  % table support for pandoc > 1.12.2
    \usepackage[inline]{enumitem} % IRkernel/repr support (it uses the enumerate* environment)
    \usepackage[normalem]{ulem} % ulem is needed to support strikethroughs (\sout)
                                % normalem makes italics be italics, not underlines
    

    
    
    % Colors for the hyperref package
    \definecolor{urlcolor}{rgb}{0,.145,.698}
    \definecolor{linkcolor}{rgb}{.71,0.21,0.01}
    \definecolor{citecolor}{rgb}{.12,.54,.11}

    % ANSI colors
    \definecolor{ansi-black}{HTML}{3E424D}
    \definecolor{ansi-black-intense}{HTML}{282C36}
    \definecolor{ansi-red}{HTML}{E75C58}
    \definecolor{ansi-red-intense}{HTML}{B22B31}
    \definecolor{ansi-green}{HTML}{00A250}
    \definecolor{ansi-green-intense}{HTML}{007427}
    \definecolor{ansi-yellow}{HTML}{DDB62B}
    \definecolor{ansi-yellow-intense}{HTML}{B27D12}
    \definecolor{ansi-blue}{HTML}{208FFB}
    \definecolor{ansi-blue-intense}{HTML}{0065CA}
    \definecolor{ansi-magenta}{HTML}{D160C4}
    \definecolor{ansi-magenta-intense}{HTML}{A03196}
    \definecolor{ansi-cyan}{HTML}{60C6C8}
    \definecolor{ansi-cyan-intense}{HTML}{258F8F}
    \definecolor{ansi-white}{HTML}{C5C1B4}
    \definecolor{ansi-white-intense}{HTML}{A1A6B2}

    % commands and environments needed by pandoc snippets
    % extracted from the output of `pandoc -s`
    \providecommand{\tightlist}{%
      \setlength{\itemsep}{0pt}\setlength{\parskip}{0pt}}
    \DefineVerbatimEnvironment{Highlighting}{Verbatim}{commandchars=\\\{\}}
    % Add ',fontsize=\small' for more characters per line
    \newenvironment{Shaded}{}{}
    \newcommand{\KeywordTok}[1]{\textcolor[rgb]{0.00,0.44,0.13}{\textbf{{#1}}}}
    \newcommand{\DataTypeTok}[1]{\textcolor[rgb]{0.56,0.13,0.00}{{#1}}}
    \newcommand{\DecValTok}[1]{\textcolor[rgb]{0.25,0.63,0.44}{{#1}}}
    \newcommand{\BaseNTok}[1]{\textcolor[rgb]{0.25,0.63,0.44}{{#1}}}
    \newcommand{\FloatTok}[1]{\textcolor[rgb]{0.25,0.63,0.44}{{#1}}}
    \newcommand{\CharTok}[1]{\textcolor[rgb]{0.25,0.44,0.63}{{#1}}}
    \newcommand{\StringTok}[1]{\textcolor[rgb]{0.25,0.44,0.63}{{#1}}}
    \newcommand{\CommentTok}[1]{\textcolor[rgb]{0.38,0.63,0.69}{\textit{{#1}}}}
    \newcommand{\OtherTok}[1]{\textcolor[rgb]{0.00,0.44,0.13}{{#1}}}
    \newcommand{\AlertTok}[1]{\textcolor[rgb]{1.00,0.00,0.00}{\textbf{{#1}}}}
    \newcommand{\FunctionTok}[1]{\textcolor[rgb]{0.02,0.16,0.49}{{#1}}}
    \newcommand{\RegionMarkerTok}[1]{{#1}}
    \newcommand{\ErrorTok}[1]{\textcolor[rgb]{1.00,0.00,0.00}{\textbf{{#1}}}}
    \newcommand{\NormalTok}[1]{{#1}}
    
    % Additional commands for more recent versions of Pandoc
    \newcommand{\ConstantTok}[1]{\textcolor[rgb]{0.53,0.00,0.00}{{#1}}}
    \newcommand{\SpecialCharTok}[1]{\textcolor[rgb]{0.25,0.44,0.63}{{#1}}}
    \newcommand{\VerbatimStringTok}[1]{\textcolor[rgb]{0.25,0.44,0.63}{{#1}}}
    \newcommand{\SpecialStringTok}[1]{\textcolor[rgb]{0.73,0.40,0.53}{{#1}}}
    \newcommand{\ImportTok}[1]{{#1}}
    \newcommand{\DocumentationTok}[1]{\textcolor[rgb]{0.73,0.13,0.13}{\textit{{#1}}}}
    \newcommand{\AnnotationTok}[1]{\textcolor[rgb]{0.38,0.63,0.69}{\textbf{\textit{{#1}}}}}
    \newcommand{\CommentVarTok}[1]{\textcolor[rgb]{0.38,0.63,0.69}{\textbf{\textit{{#1}}}}}
    \newcommand{\VariableTok}[1]{\textcolor[rgb]{0.10,0.09,0.49}{{#1}}}
    \newcommand{\ControlFlowTok}[1]{\textcolor[rgb]{0.00,0.44,0.13}{\textbf{{#1}}}}
    \newcommand{\OperatorTok}[1]{\textcolor[rgb]{0.40,0.40,0.40}{{#1}}}
    \newcommand{\BuiltInTok}[1]{{#1}}
    \newcommand{\ExtensionTok}[1]{{#1}}
    \newcommand{\PreprocessorTok}[1]{\textcolor[rgb]{0.74,0.48,0.00}{{#1}}}
    \newcommand{\AttributeTok}[1]{\textcolor[rgb]{0.49,0.56,0.16}{{#1}}}
    \newcommand{\InformationTok}[1]{\textcolor[rgb]{0.38,0.63,0.69}{\textbf{\textit{{#1}}}}}
    \newcommand{\WarningTok}[1]{\textcolor[rgb]{0.38,0.63,0.69}{\textbf{\textit{{#1}}}}}
    
    
    % Define a nice break command that doesn't care if a line doesn't already
    % exist.
    \def\br{\hspace*{\fill} \\* }
    % Math Jax compatability definitions
    \def\gt{>}
    \def\lt{<}
    % Document parameters
    \title{The Perfect Marshmallow}
    
    
    

    % Pygments definitions
    
\makeatletter
\def\PY@reset{\let\PY@it=\relax \let\PY@bf=\relax%
    \let\PY@ul=\relax \let\PY@tc=\relax%
    \let\PY@bc=\relax \let\PY@ff=\relax}
\def\PY@tok#1{\csname PY@tok@#1\endcsname}
\def\PY@toks#1+{\ifx\relax#1\empty\else%
    \PY@tok{#1}\expandafter\PY@toks\fi}
\def\PY@do#1{\PY@bc{\PY@tc{\PY@ul{%
    \PY@it{\PY@bf{\PY@ff{#1}}}}}}}
\def\PY#1#2{\PY@reset\PY@toks#1+\relax+\PY@do{#2}}

\expandafter\def\csname PY@tok@w\endcsname{\def\PY@tc##1{\textcolor[rgb]{0.73,0.73,0.73}{##1}}}
\expandafter\def\csname PY@tok@c\endcsname{\let\PY@it=\textit\def\PY@tc##1{\textcolor[rgb]{0.25,0.50,0.50}{##1}}}
\expandafter\def\csname PY@tok@cp\endcsname{\def\PY@tc##1{\textcolor[rgb]{0.74,0.48,0.00}{##1}}}
\expandafter\def\csname PY@tok@k\endcsname{\let\PY@bf=\textbf\def\PY@tc##1{\textcolor[rgb]{0.00,0.50,0.00}{##1}}}
\expandafter\def\csname PY@tok@kp\endcsname{\def\PY@tc##1{\textcolor[rgb]{0.00,0.50,0.00}{##1}}}
\expandafter\def\csname PY@tok@kt\endcsname{\def\PY@tc##1{\textcolor[rgb]{0.69,0.00,0.25}{##1}}}
\expandafter\def\csname PY@tok@o\endcsname{\def\PY@tc##1{\textcolor[rgb]{0.40,0.40,0.40}{##1}}}
\expandafter\def\csname PY@tok@ow\endcsname{\let\PY@bf=\textbf\def\PY@tc##1{\textcolor[rgb]{0.67,0.13,1.00}{##1}}}
\expandafter\def\csname PY@tok@nb\endcsname{\def\PY@tc##1{\textcolor[rgb]{0.00,0.50,0.00}{##1}}}
\expandafter\def\csname PY@tok@nf\endcsname{\def\PY@tc##1{\textcolor[rgb]{0.00,0.00,1.00}{##1}}}
\expandafter\def\csname PY@tok@nc\endcsname{\let\PY@bf=\textbf\def\PY@tc##1{\textcolor[rgb]{0.00,0.00,1.00}{##1}}}
\expandafter\def\csname PY@tok@nn\endcsname{\let\PY@bf=\textbf\def\PY@tc##1{\textcolor[rgb]{0.00,0.00,1.00}{##1}}}
\expandafter\def\csname PY@tok@ne\endcsname{\let\PY@bf=\textbf\def\PY@tc##1{\textcolor[rgb]{0.82,0.25,0.23}{##1}}}
\expandafter\def\csname PY@tok@nv\endcsname{\def\PY@tc##1{\textcolor[rgb]{0.10,0.09,0.49}{##1}}}
\expandafter\def\csname PY@tok@no\endcsname{\def\PY@tc##1{\textcolor[rgb]{0.53,0.00,0.00}{##1}}}
\expandafter\def\csname PY@tok@nl\endcsname{\def\PY@tc##1{\textcolor[rgb]{0.63,0.63,0.00}{##1}}}
\expandafter\def\csname PY@tok@ni\endcsname{\let\PY@bf=\textbf\def\PY@tc##1{\textcolor[rgb]{0.60,0.60,0.60}{##1}}}
\expandafter\def\csname PY@tok@na\endcsname{\def\PY@tc##1{\textcolor[rgb]{0.49,0.56,0.16}{##1}}}
\expandafter\def\csname PY@tok@nt\endcsname{\let\PY@bf=\textbf\def\PY@tc##1{\textcolor[rgb]{0.00,0.50,0.00}{##1}}}
\expandafter\def\csname PY@tok@nd\endcsname{\def\PY@tc##1{\textcolor[rgb]{0.67,0.13,1.00}{##1}}}
\expandafter\def\csname PY@tok@s\endcsname{\def\PY@tc##1{\textcolor[rgb]{0.73,0.13,0.13}{##1}}}
\expandafter\def\csname PY@tok@sd\endcsname{\let\PY@it=\textit\def\PY@tc##1{\textcolor[rgb]{0.73,0.13,0.13}{##1}}}
\expandafter\def\csname PY@tok@si\endcsname{\let\PY@bf=\textbf\def\PY@tc##1{\textcolor[rgb]{0.73,0.40,0.53}{##1}}}
\expandafter\def\csname PY@tok@se\endcsname{\let\PY@bf=\textbf\def\PY@tc##1{\textcolor[rgb]{0.73,0.40,0.13}{##1}}}
\expandafter\def\csname PY@tok@sr\endcsname{\def\PY@tc##1{\textcolor[rgb]{0.73,0.40,0.53}{##1}}}
\expandafter\def\csname PY@tok@ss\endcsname{\def\PY@tc##1{\textcolor[rgb]{0.10,0.09,0.49}{##1}}}
\expandafter\def\csname PY@tok@sx\endcsname{\def\PY@tc##1{\textcolor[rgb]{0.00,0.50,0.00}{##1}}}
\expandafter\def\csname PY@tok@m\endcsname{\def\PY@tc##1{\textcolor[rgb]{0.40,0.40,0.40}{##1}}}
\expandafter\def\csname PY@tok@gh\endcsname{\let\PY@bf=\textbf\def\PY@tc##1{\textcolor[rgb]{0.00,0.00,0.50}{##1}}}
\expandafter\def\csname PY@tok@gu\endcsname{\let\PY@bf=\textbf\def\PY@tc##1{\textcolor[rgb]{0.50,0.00,0.50}{##1}}}
\expandafter\def\csname PY@tok@gd\endcsname{\def\PY@tc##1{\textcolor[rgb]{0.63,0.00,0.00}{##1}}}
\expandafter\def\csname PY@tok@gi\endcsname{\def\PY@tc##1{\textcolor[rgb]{0.00,0.63,0.00}{##1}}}
\expandafter\def\csname PY@tok@gr\endcsname{\def\PY@tc##1{\textcolor[rgb]{1.00,0.00,0.00}{##1}}}
\expandafter\def\csname PY@tok@ge\endcsname{\let\PY@it=\textit}
\expandafter\def\csname PY@tok@gs\endcsname{\let\PY@bf=\textbf}
\expandafter\def\csname PY@tok@gp\endcsname{\let\PY@bf=\textbf\def\PY@tc##1{\textcolor[rgb]{0.00,0.00,0.50}{##1}}}
\expandafter\def\csname PY@tok@go\endcsname{\def\PY@tc##1{\textcolor[rgb]{0.53,0.53,0.53}{##1}}}
\expandafter\def\csname PY@tok@gt\endcsname{\def\PY@tc##1{\textcolor[rgb]{0.00,0.27,0.87}{##1}}}
\expandafter\def\csname PY@tok@err\endcsname{\def\PY@bc##1{\setlength{\fboxsep}{0pt}\fcolorbox[rgb]{1.00,0.00,0.00}{1,1,1}{\strut ##1}}}
\expandafter\def\csname PY@tok@kc\endcsname{\let\PY@bf=\textbf\def\PY@tc##1{\textcolor[rgb]{0.00,0.50,0.00}{##1}}}
\expandafter\def\csname PY@tok@kd\endcsname{\let\PY@bf=\textbf\def\PY@tc##1{\textcolor[rgb]{0.00,0.50,0.00}{##1}}}
\expandafter\def\csname PY@tok@kn\endcsname{\let\PY@bf=\textbf\def\PY@tc##1{\textcolor[rgb]{0.00,0.50,0.00}{##1}}}
\expandafter\def\csname PY@tok@kr\endcsname{\let\PY@bf=\textbf\def\PY@tc##1{\textcolor[rgb]{0.00,0.50,0.00}{##1}}}
\expandafter\def\csname PY@tok@bp\endcsname{\def\PY@tc##1{\textcolor[rgb]{0.00,0.50,0.00}{##1}}}
\expandafter\def\csname PY@tok@fm\endcsname{\def\PY@tc##1{\textcolor[rgb]{0.00,0.00,1.00}{##1}}}
\expandafter\def\csname PY@tok@vc\endcsname{\def\PY@tc##1{\textcolor[rgb]{0.10,0.09,0.49}{##1}}}
\expandafter\def\csname PY@tok@vg\endcsname{\def\PY@tc##1{\textcolor[rgb]{0.10,0.09,0.49}{##1}}}
\expandafter\def\csname PY@tok@vi\endcsname{\def\PY@tc##1{\textcolor[rgb]{0.10,0.09,0.49}{##1}}}
\expandafter\def\csname PY@tok@vm\endcsname{\def\PY@tc##1{\textcolor[rgb]{0.10,0.09,0.49}{##1}}}
\expandafter\def\csname PY@tok@sa\endcsname{\def\PY@tc##1{\textcolor[rgb]{0.73,0.13,0.13}{##1}}}
\expandafter\def\csname PY@tok@sb\endcsname{\def\PY@tc##1{\textcolor[rgb]{0.73,0.13,0.13}{##1}}}
\expandafter\def\csname PY@tok@sc\endcsname{\def\PY@tc##1{\textcolor[rgb]{0.73,0.13,0.13}{##1}}}
\expandafter\def\csname PY@tok@dl\endcsname{\def\PY@tc##1{\textcolor[rgb]{0.73,0.13,0.13}{##1}}}
\expandafter\def\csname PY@tok@s2\endcsname{\def\PY@tc##1{\textcolor[rgb]{0.73,0.13,0.13}{##1}}}
\expandafter\def\csname PY@tok@sh\endcsname{\def\PY@tc##1{\textcolor[rgb]{0.73,0.13,0.13}{##1}}}
\expandafter\def\csname PY@tok@s1\endcsname{\def\PY@tc##1{\textcolor[rgb]{0.73,0.13,0.13}{##1}}}
\expandafter\def\csname PY@tok@mb\endcsname{\def\PY@tc##1{\textcolor[rgb]{0.40,0.40,0.40}{##1}}}
\expandafter\def\csname PY@tok@mf\endcsname{\def\PY@tc##1{\textcolor[rgb]{0.40,0.40,0.40}{##1}}}
\expandafter\def\csname PY@tok@mh\endcsname{\def\PY@tc##1{\textcolor[rgb]{0.40,0.40,0.40}{##1}}}
\expandafter\def\csname PY@tok@mi\endcsname{\def\PY@tc##1{\textcolor[rgb]{0.40,0.40,0.40}{##1}}}
\expandafter\def\csname PY@tok@il\endcsname{\def\PY@tc##1{\textcolor[rgb]{0.40,0.40,0.40}{##1}}}
\expandafter\def\csname PY@tok@mo\endcsname{\def\PY@tc##1{\textcolor[rgb]{0.40,0.40,0.40}{##1}}}
\expandafter\def\csname PY@tok@ch\endcsname{\let\PY@it=\textit\def\PY@tc##1{\textcolor[rgb]{0.25,0.50,0.50}{##1}}}
\expandafter\def\csname PY@tok@cm\endcsname{\let\PY@it=\textit\def\PY@tc##1{\textcolor[rgb]{0.25,0.50,0.50}{##1}}}
\expandafter\def\csname PY@tok@cpf\endcsname{\let\PY@it=\textit\def\PY@tc##1{\textcolor[rgb]{0.25,0.50,0.50}{##1}}}
\expandafter\def\csname PY@tok@c1\endcsname{\let\PY@it=\textit\def\PY@tc##1{\textcolor[rgb]{0.25,0.50,0.50}{##1}}}
\expandafter\def\csname PY@tok@cs\endcsname{\let\PY@it=\textit\def\PY@tc##1{\textcolor[rgb]{0.25,0.50,0.50}{##1}}}

\def\PYZbs{\char`\\}
\def\PYZus{\char`\_}
\def\PYZob{\char`\{}
\def\PYZcb{\char`\}}
\def\PYZca{\char`\^}
\def\PYZam{\char`\&}
\def\PYZlt{\char`\<}
\def\PYZgt{\char`\>}
\def\PYZsh{\char`\#}
\def\PYZpc{\char`\%}
\def\PYZdl{\char`\$}
\def\PYZhy{\char`\-}
\def\PYZsq{\char`\'}
\def\PYZdq{\char`\"}
\def\PYZti{\char`\~}
% for compatibility with earlier versions
\def\PYZat{@}
\def\PYZlb{[}
\def\PYZrb{]}
\makeatother


    % Exact colors from NB
    \definecolor{incolor}{rgb}{0.0, 0.0, 0.5}
    \definecolor{outcolor}{rgb}{0.545, 0.0, 0.0}



    
    % Prevent overflowing lines due to hard-to-break entities
    \sloppy 
    % Setup hyperref package
    \hypersetup{
      breaklinks=true,  % so long urls are correctly broken across lines
      colorlinks=true,
      urlcolor=urlcolor,
      linkcolor=linkcolor,
      citecolor=citecolor,
      }
    % Slightly bigger margins than the latex defaults
    
    \geometry{verbose,tmargin=1in,bmargin=1in,lmargin=1in,rmargin=1in}
    
    

    \begin{document}
    
    
    \maketitle
    
    

    
    \hypertarget{modeling-the-perfectly-roasted-marshmallow}{%
\section{Modeling the Perfectly Roasted
Marshmallow}\label{modeling-the-perfectly-roasted-marshmallow}}

Leila Merzenich and Nathan Weil

    \hypertarget{what-temperature-and-amount-of-time-will-create-the-perfectly-roasted-marshmallow}{%
\subsubsection{What temperature and amount of time will create the
perfectly roasted
marshmallow?}\label{what-temperature-and-amount-of-time-will-create-the-perfectly-roasted-marshmallow}}

    \begin{Verbatim}[commandchars=\\\{\}]
{\color{incolor}In [{\color{incolor}1}]:} \PY{c+c1}{\PYZsh{} Configure Jupyter so figures appear in the notebook}
        \PY{o}{\PYZpc{}}\PY{k}{matplotlib} inline
        
        \PY{c+c1}{\PYZsh{} Configure Jupyter to display the assigned value after an assignment}
        \PY{o}{\PYZpc{}}\PY{k}{config} InteractiveShell.ast\PYZus{}node\PYZus{}interactivity=\PYZsq{}last\PYZus{}expr\PYZus{}or\PYZus{}assign\PYZsq{}
        
        \PY{c+c1}{\PYZsh{} Import functions from the modsim.py module}
        \PY{k+kn}{from} \PY{n+nn}{modsim} \PY{k}{import} \PY{o}{*}
        
        \PY{c+c1}{\PYZsh{} Import numpy}
        \PY{k+kn}{import} \PY{n+nn}{numpy}
\end{Verbatim}


    To answer this question, we need to see at what temperature and time the
core of the marshmallow is roasted thoroughly and the outside is crispy
but not burnt. In order to understand the heat transfer process through
the marshmallow from the outside air, we separated the marshmallow into
three concentric sections with the same masses and varying thicknesses.
This simplification is helpful because it is challenging to model the
marshmallow as one large flow of energy, and does not over-simplify the
heat transfer process.

The energy held in each subsection is a stock, as well as the
temperature that the amount of energy creates, because the temperatures
and energies all affect each other and need to be calculated over time.
The mass of each ring in the marshmallow is the same, and the thickness
and surface area were calculated based on that. We used the conductivity
of bread as a proxy for that of a marshmallow, which is a big assumption
to make but is reasonable given that they are similar consistencies and
we could not find the conductivity of a marshmallow, but found a
reasonable range of possible values and that of bread fit within our
range. We were able to find the specific heat of a marshmallow.

The initial temperature of our marshmallow is 50 degrees fahrenheit or
283 Kelvin, an average temperature for a night camping. Using this we
calculated the initial heat energy stored in each layer of the
marshmallow below.

    \begin{Verbatim}[commandchars=\\\{\}]
{\color{incolor}In [{\color{incolor}2}]:} \PY{c+c1}{\PYZsh{} Sets external temperature for example}
        \PY{n}{eTemp}\PY{o}{=}\PY{l+m+mi}{360}
        
        \PY{c+c1}{\PYZsh{} Sets initial temperature and calculates initial energy}
        \PY{n}{initial\PYZus{}temp} \PY{o}{=} \PY{l+m+mi}{283}
        \PY{n}{initial\PYZus{}energy} \PY{o}{=} \PY{n}{initial\PYZus{}temp}\PY{o}{*}\PY{p}{(}\PY{l+m+mf}{2.33}\PY{o}{*}\PY{l+m+mf}{2.02}\PY{p}{)}
\end{Verbatim}


\begin{Verbatim}[commandchars=\\\{\}]
{\color{outcolor}Out[{\color{outcolor}2}]:} 1331.9678
\end{Verbatim}
            
    \begin{Verbatim}[commandchars=\\\{\}]
{\color{incolor}In [{\color{incolor}3}]:} \PY{k}{def} \PY{n+nf}{make\PYZus{}system}\PY{p}{(}\PY{n}{eTemp}\PY{p}{)}\PY{p}{:}
            \PY{l+s+sd}{\PYZdq{}\PYZdq{}\PYZdq{}Makes a system object including all parameters}
        \PY{l+s+sd}{        }
        \PY{l+s+sd}{    eTemp: cooking temperature outside of marshmallow}
        \PY{l+s+sd}{    }
        \PY{l+s+sd}{    returns: System object}
        \PY{l+s+sd}{    \PYZdq{}\PYZdq{}\PYZdq{}}
           
            \PY{c+c1}{\PYZsh{} Initializes state object}
            \PY{n}{init} \PY{o}{=} \PY{n}{State}\PY{p}{(}\PY{n}{outer}\PY{o}{=}\PY{n}{initial\PYZus{}energy}\PY{p}{,} \PY{n}{middle}\PY{o}{=}\PY{n}{initial\PYZus{}energy}\PY{p}{,} \PY{n}{inner}\PY{o}{=}\PY{n}{initial\PYZus{}energy}\PY{p}{,} 
                         \PY{n}{oTemp}\PY{o}{=}\PY{n}{initial\PYZus{}temp}\PY{p}{,} \PY{n}{mTemp}\PY{o}{=}\PY{n}{initial\PYZus{}temp}\PY{p}{,} \PY{n}{iTemp}\PY{o}{=}\PY{n}{initial\PYZus{}temp}\PY{p}{,} \PY{n}{o\PYZus{}proportion}\PY{o}{=}\PY{l+m+mi}{0}\PY{p}{,} \PY{n}{i\PYZus{}proportion}\PY{o}{=}\PY{l+m+mi}{0}\PY{p}{)}
            
            \PY{c+c1}{\PYZsh{} Size of the time steps}
            \PY{n}{dt}\PY{o}{=}\PY{l+m+mi}{10} 
            \PY{c+c1}{\PYZsh{} Thermal conductivity}
            \PY{n}{conductivity}\PY{o}{=}\PY{l+m+mf}{0.1} 
            \PY{c+c1}{\PYZsh{} Surface are of a marshmallow}
            \PY{n}{areaOuter}\PY{o}{=}\PY{l+m+mf}{0.004054} 
            \PY{c+c1}{\PYZsh{} Surface area of the middle section of the marshmallow}
            \PY{n}{areaMiddle}\PY{o}{=}\PY{l+m+mf}{0.003091} 
            \PY{c+c1}{\PYZsh{} Surface are of the inner section}
            \PY{n}{areaInner}\PY{o}{=}\PY{l+m+mf}{0.001946} 
            \PY{c+c1}{\PYZsh{} Thickness of the outer layer}
            \PY{n}{thicknessOuter}\PY{o}{=}\PY{l+m+mf}{0.00161} 
            \PY{c+c1}{\PYZsh{} Thickness of the middle layer}
            \PY{n}{thicknessMiddle}\PY{o}{=}\PY{l+m+mf}{0.00229} 
            \PY{c+c1}{\PYZsh{} Thickness of the inner core}
            \PY{n}{thicknessInner}\PY{o}{=}\PY{l+m+mf}{0.0088} 
            \PY{c+c1}{\PYZsh{} Each layer has the same mass}
            \PY{n}{mass}\PY{o}{=}\PY{l+m+mf}{2.33}
            \PY{c+c1}{\PYZsh{} Specific heat capacity of a marshmallow}
            \PY{n}{specificMarshmallow}\PY{o}{=}\PY{l+m+mf}{2.02} 
            \PY{c+c1}{\PYZsh{} Initial time}
            \PY{n}{t0}\PY{o}{=}\PY{l+m+mi}{0}
            \PY{c+c1}{\PYZsh{} Ending time in seconds}
            \PY{n}{t\PYZus{}end}\PY{o}{=}\PY{l+m+mi}{1000} 
            \PY{c+c1}{\PYZsh{} Starting sweep temperature}
            \PY{n}{start\PYZus{}temp} \PY{o}{=} \PY{l+m+mi}{350}
            \PY{c+c1}{\PYZsh{} Ending sweep Temperature}
            \PY{n}{end\PYZus{}temp} \PY{o}{=} \PY{l+m+mi}{450}
        
            \PY{k}{return} \PY{n}{System}\PY{p}{(}\PY{n}{init}\PY{o}{=}\PY{n}{init}\PY{p}{,} \PY{n}{dt}\PY{o}{=}\PY{n}{dt}\PY{p}{,} \PY{n}{conductivity}\PY{o}{=}\PY{n}{conductivity}\PY{p}{,} 
                          \PY{n}{areaOuter}\PY{o}{=}\PY{n}{areaOuter}\PY{p}{,} \PY{n}{areaMiddle}\PY{o}{=}\PY{n}{areaMiddle}\PY{p}{,} 
                          \PY{n}{areaInner}\PY{o}{=}\PY{n}{areaInner}\PY{p}{,} \PY{n}{thicknessOuter}\PY{o}{=}\PY{n}{thicknessOuter}\PY{p}{,} 
                          \PY{n}{thicknessMiddle}\PY{o}{=}\PY{n}{thicknessMiddle}\PY{p}{,} 
                          \PY{n}{thicknessInner}\PY{o}{=}\PY{n}{thicknessInner}\PY{p}{,} \PY{n}{mass}\PY{o}{=}\PY{n}{mass}\PY{p}{,} 
                          \PY{n}{specificMarshmallow}\PY{o}{=}\PY{n}{specificMarshmallow}\PY{p}{,} \PY{n}{t0}\PY{o}{=}\PY{n}{t0}\PY{p}{,} \PY{n}{t\PYZus{}end}\PY{o}{=}\PY{n}{t\PYZus{}end}\PY{p}{,}
                          \PY{n}{start\PYZus{}temp}\PY{o}{=}\PY{n}{start\PYZus{}temp}\PY{p}{,} \PY{n}{end\PYZus{}temp}\PY{o}{=}\PY{n}{end\PYZus{}temp}\PY{p}{,} \PY{n}{eTemp}\PY{o}{=}\PY{n}{eTemp}\PY{p}{)}
\end{Verbatim}


    In order to calculate the heat transfered from the outside air to the
marshmallow and from each section to the inner section, we used the
following equations:

dQ/dt=(kA(T2-T1))/d

T=Q/mc

Q=heat transfered

k=thermal conductivity

A=surface area

T1=first temp

T2=second temp

d=thickness

m=mass

Cmarshmallow=specific heat

At each step, we will add the energy transfered from the next outer
section and subtract the energy transfered to the next inner section
using the temperature from the previous time step. The temperature is
recalculated at each step using the total energy during that timestep.

    \begin{Verbatim}[commandchars=\\\{\}]
{\color{incolor}In [{\color{incolor}4}]:} \PY{k}{def} \PY{n+nf}{update\PYZus{}func}\PY{p}{(}\PY{n}{state}\PY{p}{,} \PY{n}{t}\PY{p}{,} \PY{n}{system}\PY{p}{)}\PY{p}{:}
            \PY{l+s+sd}{\PYZdq{}\PYZdq{}\PYZdq{}calculates values and updates state for each time step }
        \PY{l+s+sd}{     }
        \PY{l+s+sd}{    state: State object}
        \PY{l+s+sd}{    t: current time step}
        \PY{l+s+sd}{    system: System object}
        \PY{l+s+sd}{    }
        \PY{l+s+sd}{    returns: updated State object}
        \PY{l+s+sd}{    \PYZdq{}\PYZdq{}\PYZdq{}}
            \PY{c+c1}{\PYZsh{} Initalizes local state}
            \PY{n}{outer}\PY{p}{,} \PY{n}{middle}\PY{p}{,} \PY{n}{inner}\PY{p}{,} \PY{n}{oTemp}\PY{p}{,} \PY{n}{mTemp}\PY{p}{,} \PY{n}{iTemp}\PY{p}{,} \PY{n}{o\PYZus{}proportion}\PY{p}{,} \PY{n}{i\PYZus{}proportion} \PY{o}{=} \PY{n}{state}
            
            \PY{c+c1}{\PYZsh{} Unpacks System object}
            \PY{n}{unpack}\PY{p}{(}\PY{n}{system}\PY{p}{)}
            
            \PY{c+c1}{\PYZsh{} Energy transfered from outside air to outer layer}
            \PY{n}{dQdt1} \PY{o}{=} \PY{p}{(}\PY{n}{conductivity}\PY{o}{*}\PY{n}{areaOuter}\PY{o}{*}\PY{p}{(}\PY{n}{eTemp}\PY{o}{\PYZhy{}}\PY{n}{oTemp}\PY{p}{)}\PY{p}{)}\PY{o}{/}\PY{n}{thicknessOuter} 
            \PY{c+c1}{\PYZsh{} From outer to midle layer}
            \PY{n}{dQdt2} \PY{o}{=} \PY{p}{(}\PY{n}{conductivity}\PY{o}{*}\PY{n}{areaMiddle}\PY{o}{*}\PY{p}{(}\PY{n}{oTemp}\PY{o}{\PYZhy{}}\PY{n}{mTemp}\PY{p}{)}\PY{p}{)}\PY{o}{/}\PY{n}{thicknessMiddle} 
            \PY{c+c1}{\PYZsh{} Middle to inner layer}
            \PY{n}{dQdt3} \PY{o}{=} \PY{p}{(}\PY{n}{conductivity}\PY{o}{*}\PY{n}{areaInner}\PY{o}{*}\PY{p}{(}\PY{n}{mTemp}\PY{o}{\PYZhy{}}\PY{n}{iTemp}\PY{p}{)}\PY{p}{)}\PY{o}{/}\PY{n}{thicknessInner} \PY{c+c1}{\PYZsh{}W/mK*m\PYZca{}2*K/m = W}
            
            \PY{c+c1}{\PYZsh{} Add energy transfered from air and subtract energy lost to middle layer}
            \PY{n}{outer} \PY{o}{+}\PY{o}{=} \PY{p}{(}\PY{n}{dQdt1} \PY{o}{\PYZhy{}} \PY{n}{dQdt2}\PY{p}{)}\PY{o}{*}\PY{n}{dt} 
            \PY{c+c1}{\PYZsh{} Add energy from outer and subtract energy to inner}
            \PY{n}{middle} \PY{o}{+}\PY{o}{=} \PY{p}{(}\PY{n}{dQdt2} \PY{o}{\PYZhy{}} \PY{n}{dQdt3}\PY{p}{)}\PY{o}{*}\PY{n}{dt} 
            \PY{c+c1}{\PYZsh{} Add energy from middle}
            \PY{n}{inner} \PY{o}{+}\PY{o}{=} \PY{p}{(}\PY{n}{dQdt3}\PY{p}{)}\PY{o}{*}\PY{n}{dt} 
            
            \PY{c+c1}{\PYZsh{} Converts Energy to Temperature}
            \PY{c+c1}{\PYZsh{} Energy/(kg*energy/(kg*K)) = K}
            \PY{n}{iTemp} \PY{o}{=} \PY{n}{inner}\PY{o}{/}\PY{p}{(}\PY{n}{mass}\PY{o}{*}\PY{n}{specificMarshmallow}\PY{p}{)} 
            \PY{n}{mTemp} \PY{o}{=} \PY{n}{middle}\PY{o}{/}\PY{p}{(}\PY{n}{mass}\PY{o}{*}\PY{n}{specificMarshmallow}\PY{p}{)}
            \PY{n}{oTemp} \PY{o}{=} \PY{n}{outer}\PY{o}{/}\PY{p}{(}\PY{n}{mass}\PY{o}{*}\PY{n}{specificMarshmallow}\PY{p}{)}
            
            \PY{c+c1}{\PYZsh{} Calculates temperature as proportion of temperature range}
            \PY{n}{o\PYZus{}proportion} \PY{o}{=} \PY{p}{(}\PY{n}{oTemp}\PY{o}{\PYZhy{}}\PY{n}{initial\PYZus{}temp}\PY{p}{)}\PY{o}{/}\PY{p}{(}\PY{n}{end\PYZus{}temp}\PY{o}{\PYZhy{}}\PY{n}{initial\PYZus{}temp}\PY{p}{)}
            \PY{n}{i\PYZus{}proportion} \PY{o}{=} \PY{p}{(}\PY{n}{iTemp}\PY{o}{\PYZhy{}}\PY{n}{initial\PYZus{}temp}\PY{p}{)}\PY{o}{/}\PY{p}{(}\PY{n}{end\PYZus{}temp}\PY{o}{\PYZhy{}}\PY{n}{initial\PYZus{}temp}\PY{p}{)} 
         
            \PY{k}{return} \PY{n}{State}\PY{p}{(}\PY{n}{outer}\PY{o}{=}\PY{n}{outer}\PY{p}{,} \PY{n}{middle}\PY{o}{=}\PY{n}{middle}\PY{p}{,} \PY{n}{inner}\PY{o}{=}\PY{n}{inner}\PY{p}{,}
                         \PY{n}{oTemp}\PY{o}{=}\PY{n}{oTemp}\PY{p}{,} \PY{n}{mTemp}\PY{o}{=}\PY{n}{mTemp}\PY{p}{,} \PY{n}{iTemp}\PY{o}{=}\PY{n}{iTemp}\PY{p}{,} \PY{n}{o\PYZus{}proportion}\PY{o}{=}\PY{n}{o\PYZus{}proportion}\PY{p}{,}
                         \PY{n}{i\PYZus{}proportion}\PY{o}{=}\PY{n}{i\PYZus{}proportion}\PY{p}{)}
\end{Verbatim}


    \begin{Verbatim}[commandchars=\\\{\}]
{\color{incolor}In [{\color{incolor}5}]:} \PY{k}{def} \PY{n+nf}{run\PYZus{}simulation}\PY{p}{(}\PY{n}{system}\PY{p}{,} \PY{n}{update\PYZus{}func}\PY{p}{)}\PY{p}{:}
            \PY{l+s+sd}{\PYZdq{}\PYZdq{}\PYZdq{}Runs a simulation of the system.}
        \PY{l+s+sd}{        }
        \PY{l+s+sd}{    system: System object}
        \PY{l+s+sd}{    update\PYZus{}func: function that updates state}
        \PY{l+s+sd}{    }
        \PY{l+s+sd}{    returns: TimeFrame}
        \PY{l+s+sd}{    \PYZdq{}\PYZdq{}\PYZdq{}}
            \PY{c+c1}{\PYZsh{} Unpacks System object}
            \PY{n}{unpack}\PY{p}{(}\PY{n}{system}\PY{p}{)}
            
            \PY{c+c1}{\PYZsh{} Initalizes time frame}
            \PY{n}{frame} \PY{o}{=} \PY{n}{TimeFrame}\PY{p}{(}\PY{n}{columns}\PY{o}{=}\PY{n}{init}\PY{o}{.}\PY{n}{index}\PY{p}{)}
            \PY{n}{frame}\PY{o}{.}\PY{n}{row}\PY{p}{[}\PY{n}{t0}\PY{p}{]} \PY{o}{=} \PY{n}{init}
            
            \PY{c+c1}{\PYZsh{} Runs update function for each time step}
            \PY{k}{for} \PY{n}{t} \PY{o+ow}{in} \PY{n}{linrange}\PY{p}{(}\PY{n}{t0}\PY{p}{,} \PY{n}{t\PYZus{}end}\PY{p}{,} \PY{n}{dt}\PY{p}{)}\PY{p}{:}
                \PY{n}{frame}\PY{o}{.}\PY{n}{row}\PY{p}{[}\PY{n}{t}\PY{o}{+}\PY{n}{dt}\PY{p}{]} \PY{o}{=} \PY{n}{update\PYZus{}func}\PY{p}{(}\PY{n}{frame}\PY{o}{.}\PY{n}{row}\PY{p}{[}\PY{n}{t}\PY{p}{]}\PY{p}{,} \PY{n}{t}\PY{p}{,} \PY{n}{system}\PY{p}{)}
            
            \PY{k}{return} \PY{n}{frame}
\end{Verbatim}


    \begin{Verbatim}[commandchars=\\\{\}]
{\color{incolor}In [{\color{incolor}6}]:} \PY{k}{def} \PY{n+nf}{plot\PYZus{}results}\PY{p}{(}\PY{n}{oTemp}\PY{p}{,} \PY{n}{mTemp}\PY{p}{,} \PY{n}{iTemp}\PY{p}{)}\PY{p}{:}
            \PY{l+s+sd}{\PYZdq{}\PYZdq{}\PYZdq{}Plots results from run\PYZus{}simulation}
        \PY{l+s+sd}{        }
        \PY{l+s+sd}{    oTemp: outer temperature of marshmallow}
        \PY{l+s+sd}{    mTemp: middle temperature of marshmallow}
        \PY{l+s+sd}{    iTemp: inner temperature of marshmallow}
        \PY{l+s+sd}{    }
        \PY{l+s+sd}{    Displays: Plot of marshmallow section temperature over time}
        \PY{l+s+sd}{    \PYZdq{}\PYZdq{}\PYZdq{}}
            \PY{c+c1}{\PYZsh{} Plots each temperature curve}
            \PY{n}{plot}\PY{p}{(}\PY{n}{oTemp}\PY{p}{,} \PY{l+s+s1}{\PYZsq{}}\PY{l+s+s1}{\PYZhy{}\PYZhy{}}\PY{l+s+s1}{\PYZsq{}}\PY{p}{,} \PY{n}{label}\PY{o}{=}\PY{l+s+s1}{\PYZsq{}}\PY{l+s+s1}{Outer Temp}\PY{l+s+s1}{\PYZsq{}}\PY{p}{)}
            \PY{n}{plot}\PY{p}{(}\PY{n}{mTemp}\PY{p}{,} \PY{l+s+s1}{\PYZsq{}}\PY{l+s+s1}{\PYZhy{}}\PY{l+s+s1}{\PYZsq{}}\PY{p}{,} \PY{n}{label}\PY{o}{=}\PY{l+s+s1}{\PYZsq{}}\PY{l+s+s1}{Middle Temp}\PY{l+s+s1}{\PYZsq{}}\PY{p}{)}
            \PY{n}{plot}\PY{p}{(}\PY{n}{iTemp}\PY{p}{,} \PY{l+s+s1}{\PYZsq{}}\PY{l+s+s1}{\PYZhy{}}\PY{l+s+s1}{\PYZsq{}}\PY{p}{,} \PY{n}{label}\PY{o}{=}\PY{l+s+s1}{\PYZsq{}}\PY{l+s+s1}{Inner Temp}\PY{l+s+s1}{\PYZsq{}}\PY{p}{)}
            \PY{n}{decorate}\PY{p}{(}\PY{n}{xlabel}\PY{o}{=}\PY{l+s+s1}{\PYZsq{}}\PY{l+s+s1}{Time (seconds)}\PY{l+s+s1}{\PYZsq{}}\PY{p}{,}
                     \PY{n}{ylabel}\PY{o}{=}\PY{l+s+s1}{\PYZsq{}}\PY{l+s+s1}{Tempurature(K)}\PY{l+s+s1}{\PYZsq{}}\PY{p}{,}
                     \PY{n}{title}\PY{o}{=}\PY{l+s+s1}{\PYZsq{}}\PY{l+s+s1}{Temperatures of Marshmallow Zones vs Time}\PY{l+s+s1}{\PYZsq{}}\PY{p}{)}
\end{Verbatim}


    \begin{Verbatim}[commandchars=\\\{\}]
{\color{incolor}In [{\color{incolor}7}]:} \PY{c+c1}{\PYZsh{} Makes system and runs simulation}
        \PY{n}{system} \PY{o}{=} \PY{n}{make\PYZus{}system}\PY{p}{(}\PY{n}{eTemp}\PY{p}{)}
        \PY{n}{results} \PY{o}{=} \PY{n}{run\PYZus{}simulation}\PY{p}{(}\PY{n}{system}\PY{p}{,} \PY{n}{update\PYZus{}func}\PY{p}{)}
        
        \PY{n}{plot\PYZus{}results}\PY{p}{(}\PY{n}{results}\PY{o}{.}\PY{n}{oTemp}\PY{p}{,} \PY{n}{results}\PY{o}{.}\PY{n}{mTemp}\PY{p}{,} \PY{n}{results}\PY{o}{.}\PY{n}{iTemp}\PY{p}{)}
\end{Verbatim}


    \begin{center}
    \adjustimage{max size={0.9\linewidth}{0.9\paperheight}}{output_10_0.png}
    \end{center}
    { \hspace*{\fill} \\}
    
    The graph above shows the temperatures of the three subsections of the
marshmallow over time, with an external temperature of 360 Kelvin. The
conduction of heat through the different layers of the marshmallow is
evident by the differing slopes of the three subsections.

    In order to show our model's results in a comprehensible manner, we
created a heat map using a sweep of time and external temperature. We
wanted to do this with the core temperature as well as the outer
temperature to understand at what values of both the marshmallow would
be perfect.

Then, given that the internal temperature must be at least 336 Kelvin
and the external temperature cannot exceed 358 Kelvin, we created a
ideal heat zone confined by this upper limit of the outer section and
this lower limit of the inner section of the marshmallow.

    \begin{Verbatim}[commandchars=\\\{\}]
{\color{incolor}In [{\color{incolor}8}]:} \PY{k}{def} \PY{n+nf}{sweep\PYZus{}params}\PY{p}{(}\PY{n}{system}\PY{p}{)}\PY{p}{:}
            \PY{l+s+sd}{\PYZdq{}\PYZdq{}\PYZdq{}}
        \PY{l+s+sd}{    Sweeps external temperature, running simulation each time}
        \PY{l+s+sd}{    Saves data in arrays for ploting on heat map}
        \PY{l+s+sd}{        }
        \PY{l+s+sd}{    system: System object}
        \PY{l+s+sd}{    }
        \PY{l+s+sd}{    returns: array of outer results, }
        \PY{l+s+sd}{             array of inner results, }
        \PY{l+s+sd}{             array of ideal zones}
        \PY{l+s+sd}{    \PYZdq{}\PYZdq{}\PYZdq{}}
            \PY{c+c1}{\PYZsh{} Initalizes external temperature sweep values}
            \PY{n}{sweep\PYZus{}array} \PY{o}{=} \PY{n}{linrange}\PY{p}{(}\PY{l+m+mi}{355}\PY{p}{,} \PY{l+m+mi}{400}\PY{p}{,} \PY{o}{.}\PY{l+m+mi}{5}\PY{p}{)}
        
            \PY{c+c1}{\PYZsh{} Initializes arrays to return}
            \PY{n}{o\PYZus{}results} \PY{o}{=} \PY{p}{[}\PY{p}{]}
            \PY{n}{i\PYZus{}results} \PY{o}{=} \PY{p}{[}\PY{p}{]}
            \PY{n}{zone} \PY{o}{=} \PY{p}{[}\PY{p}{]}
            
            \PY{c+c1}{\PYZsh{} Sweeps external temperature values}
            \PY{k}{for} \PY{n}{eTemp} \PY{o+ow}{in} \PY{n}{sweep\PYZus{}array}\PY{p}{:}
        
                \PY{c+c1}{\PYZsh{} Makes new system with external temperature from sweep}
                \PY{n}{system} \PY{o}{=} \PY{n}{make\PYZus{}system}\PY{p}{(}\PY{n}{eTemp}\PY{p}{)}
                
                \PY{c+c1}{\PYZsh{} Runs simulation for new system}
                \PY{n}{data} \PY{o}{=} \PY{n}{run\PYZus{}simulation}\PY{p}{(}\PY{n}{system}\PY{p}{,} \PY{n}{update\PYZus{}func}\PY{p}{)}
                
                \PY{c+c1}{\PYZsh{} Sets step arrays as calculated proportions from run\PYZus{}simulation}
                \PY{c+c1}{\PYZsh{} Negative values to invert heat map colors}
                \PY{n}{o\PYZus{}step} \PY{o}{=} \PY{o}{\PYZhy{}}\PY{n}{data}\PY{o}{.}\PY{n}{o\PYZus{}proportion}
                \PY{n}{i\PYZus{}step} \PY{o}{=} \PY{o}{\PYZhy{}}\PY{n}{data}\PY{o}{.}\PY{n}{i\PYZus{}proportion}
                
                \PY{c+c1}{\PYZsh{} Initializes zone step array}
                \PY{n}{zone\PYZus{}step} \PY{o}{=} \PY{p}{[}\PY{p}{]}
                        
                \PY{c+c1}{\PYZsh{} Sweeps internal and outer temperatures to find perfect range}
                \PY{k}{for} \PY{n}{t} \PY{o+ow}{in} \PY{n}{linrange}\PY{p}{(}\PY{n}{t0}\PY{p}{,} \PY{n}{t\PYZus{}end}\PY{p}{,} \PY{n}{dt}\PY{p}{)}\PY{p}{:}
                    \PY{k}{if} \PY{n}{data}\PY{o}{.}\PY{n}{iTemp}\PY{p}{[}\PY{n}{t}\PY{p}{]} \PY{o}{\PYZlt{}} \PY{l+m+mi}{336} \PY{o+ow}{and} \PY{n}{data}\PY{o}{.}\PY{n}{oTemp}\PY{p}{[}\PY{n}{t}\PY{p}{]} \PY{o}{\PYZgt{}} \PY{l+m+mi}{358}\PY{p}{:}
                        \PY{n}{zone\PYZus{}num} \PY{o}{=} \PY{o}{.}\PY{l+m+mi}{4}
                    \PY{k}{elif} \PY{n}{data}\PY{o}{.}\PY{n}{oTemp}\PY{p}{[}\PY{n}{t}\PY{p}{]} \PY{o}{\PYZgt{}} \PY{l+m+mi}{358}\PY{p}{:}
                        \PY{n}{zone\PYZus{}num} \PY{o}{=} \PY{l+m+mi}{0}
                    \PY{k}{elif} \PY{n}{data}\PY{o}{.}\PY{n}{iTemp}\PY{p}{[}\PY{n}{t}\PY{p}{]} \PY{o}{\PYZlt{}} \PY{l+m+mi}{336}\PY{p}{:}
                        \PY{n}{zone\PYZus{}num} \PY{o}{=} \PY{l+m+mi}{1}
                    \PY{k}{else}\PY{p}{:}
                        \PY{n}{zone\PYZus{}num} \PY{o}{=} \PY{l+m+mf}{0.65}
                        
                    \PY{k}{if} \PY{l+m+mf}{335.5} \PY{o}{\PYZlt{}} \PY{n}{data}\PY{o}{.}\PY{n}{iTemp}\PY{p}{[}\PY{n}{t}\PY{p}{]} \PY{o}{\PYZlt{}} \PY{l+m+mf}{336.5} \PY{o+ow}{and} \PY{l+m+mi}{357} \PY{o}{\PYZlt{}} \PY{n}{data}\PY{o}{.}\PY{n}{oTemp}\PY{p}{[}\PY{n}{t}\PY{p}{]} \PY{o}{\PYZlt{}} \PY{l+m+mf}{358.5}\PY{p}{:}
                        \PY{n+nb}{print}\PY{p}{(}\PY{l+s+s1}{\PYZsq{}}\PY{l+s+s1}{Ideal Time: }\PY{l+s+s1}{\PYZsq{}} \PY{o}{+} \PY{n+nb}{str}\PY{p}{(}\PY{n}{t}\PY{p}{)}\PY{p}{)}
                        \PY{n+nb}{print}\PY{p}{(}\PY{l+s+s1}{\PYZsq{}}\PY{l+s+s1}{Ideal External Temperature: }\PY{l+s+s1}{\PYZsq{}} \PY{o}{+} \PY{n+nb}{str}\PY{p}{(}\PY{n}{eTemp}\PY{p}{)}\PY{p}{)}
                    \PY{c+c1}{\PYZsh{} Adds result to zone\PYZus{}step array}
                    \PY{n}{zone\PYZus{}step}\PY{o}{.}\PY{n}{append}\PY{p}{(}\PY{n}{zone\PYZus{}num}\PY{p}{)}
                
                \PY{c+c1}{\PYZsh{} Updates all main arrays}
                \PY{n}{o\PYZus{}results}\PY{o}{.}\PY{n}{append}\PY{p}{(}\PY{n}{o\PYZus{}step}\PY{p}{)}
                \PY{n}{i\PYZus{}results}\PY{o}{.}\PY{n}{append}\PY{p}{(}\PY{n}{i\PYZus{}step}\PY{p}{)}
                \PY{n}{zone}\PY{o}{.}\PY{n}{append}\PY{p}{(}\PY{n}{zone\PYZus{}step}\PY{p}{)}
                
            \PY{k}{return} \PY{n}{o\PYZus{}results}\PY{p}{,} \PY{n}{i\PYZus{}results}\PY{p}{,} \PY{n}{zone}
\end{Verbatim}


    \begin{Verbatim}[commandchars=\\\{\}]
{\color{incolor}In [{\color{incolor}9}]:} \PY{k}{def} \PY{n+nf}{map\PYZus{}values}\PY{p}{(}\PY{n}{results}\PY{p}{,} \PY{n}{title}\PY{p}{)}\PY{p}{:}
            \PY{l+s+sd}{\PYZdq{}\PYZdq{}\PYZdq{} Maps arrays on to heat maps}
        \PY{l+s+sd}{        }
        \PY{l+s+sd}{    results: single array of values from sweep\PYZus{}params}
        \PY{l+s+sd}{    title: title of heat map}
        \PY{l+s+sd}{    }
        \PY{l+s+sd}{    Displays: heat map of data stored in results}
        \PY{l+s+sd}{    \PYZdq{}\PYZdq{}\PYZdq{}}    
            \PY{c+c1}{\PYZsh{} Labels heat map}
            \PY{n}{decorate}\PY{p}{(}\PY{n}{xlabel}\PY{o}{=}\PY{l+s+s1}{\PYZsq{}}\PY{l+s+s1}{Time (deca\PYZhy{}seconds)}\PY{l+s+s1}{\PYZsq{}}\PY{p}{,}
                 \PY{n}{ylabel}\PY{o}{=}\PY{l+s+s1}{\PYZsq{}}\PY{l+s+s1}{Relative Tempurature}\PY{l+s+s1}{\PYZsq{}}\PY{p}{,}
                 \PY{n}{title}\PY{o}{=}\PY{n}{title}\PY{p}{)}
                     
            \PY{c+c1}{\PYZsh{} Plots heat map}
            \PY{n}{plt}\PY{o}{.}\PY{n}{imshow}\PY{p}{(}\PY{n}{results}\PY{p}{,} \PY{n}{cmap}\PY{o}{=}\PY{l+s+s1}{\PYZsq{}}\PY{l+s+s1}{hot}\PY{l+s+s1}{\PYZsq{}}\PY{p}{,} \PY{n}{interpolation}\PY{o}{=}\PY{l+s+s1}{\PYZsq{}}\PY{l+s+s1}{nearest}\PY{l+s+s1}{\PYZsq{}}\PY{p}{)}
            \PY{n}{plt}\PY{o}{.}\PY{n}{show}\PY{p}{(}\PY{p}{)}
\end{Verbatim}


    \begin{Verbatim}[commandchars=\\\{\}]
{\color{incolor}In [{\color{incolor}10}]:} \PY{n}{o\PYZus{}results}\PY{p}{,} \PY{n}{i\PYZus{}results}\PY{p}{,} \PY{n}{zone} \PY{o}{=} \PY{n}{sweep\PYZus{}params}\PY{p}{(}\PY{n}{system}\PY{p}{)}
\end{Verbatim}


    \begin{Verbatim}[commandchars=\\\{\}]
Ideal Time: 390
Ideal External Temperature: 359.5
Ideal Time: 380
Ideal External Temperature: 360.0
Ideal Time: 380
Ideal External Temperature: 360.5

    \end{Verbatim}

    \begin{Verbatim}[commandchars=\\\{\}]
{\color{incolor}In [{\color{incolor}11}]:} \PY{n}{map\PYZus{}values}\PY{p}{(}\PY{n}{o\PYZus{}results}\PY{p}{,} \PY{l+s+s1}{\PYZsq{}}\PY{l+s+s1}{Outer Layer Heat Map}\PY{l+s+s1}{\PYZsq{}}\PY{p}{)}
         \PY{n}{map\PYZus{}values}\PY{p}{(}\PY{n}{i\PYZus{}results}\PY{p}{,} \PY{l+s+s1}{\PYZsq{}}\PY{l+s+s1}{Inner Layer Heat Map}\PY{l+s+s1}{\PYZsq{}}\PY{p}{)}
         \PY{n}{map\PYZus{}values}\PY{p}{(}\PY{n}{zone}\PY{p}{,} \PY{l+s+s1}{\PYZsq{}}\PY{l+s+s1}{Ideal Zone Heat Map}\PY{l+s+s1}{\PYZsq{}}\PY{p}{)}
\end{Verbatim}


    \begin{center}
    \adjustimage{max size={0.9\linewidth}{0.9\paperheight}}{output_16_0.png}
    \end{center}
    { \hspace*{\fill} \\}
    
    \begin{center}
    \adjustimage{max size={0.9\linewidth}{0.9\paperheight}}{output_16_1.png}
    \end{center}
    { \hspace*{\fill} \\}
    
    \begin{center}
    \adjustimage{max size={0.9\linewidth}{0.9\paperheight}}{output_16_2.png}
    \end{center}
    { \hspace*{\fill} \\}
    
    In the graphs above, relative temperature as it appears on the y-axis is
temperature in Kelvin from 355 to 400. Each number on the y-axis
corresponds to an increase of half a Kelvin. Each number on the x-axis
corresponds to ten seconds. Our lowest external temperature is 355
because in order to have the outside of our marshmallow be perfectly
crispy it needs to reach at minimum this temperature.

The first graph maps the external temperature, with the lightest color
being the initial temperature of the marshmallow at 283 Kelvin and the
darkest being 450 Kelvin. The second graph maps the internal temperature
on the same scale. The last graph has four sections, undercooked outside
and inside (white), burnt outside and undercooked inside (red), burnt
outside and well-cooked inside (black), and perfectly cooked outside and
in (yellow).

This graph shows that the easiest way to create the perfect marshmallow
is to keep the marshmallow in a temperature within the range of 355-358
for at least 400 seconds. However, in order to maximize time, the
marshmallow should be roasted at 360 Kelvin for 380 seconds.

    If we were to further prove our model, we should find a more substantial
way to validate our findings through further experimentation. However,
given the scope of this project, our assumptions and simplifications
were reasonable enough that I feel confident in our findings. Another
thing that would have improved our results is if we could calculate the
conductivity of a marshmallow instead of relying on bread as a proxy. In
creating this model, we spent extensive time iterating on what would be
the best way to calculate the energy stored in various parts of the
marshmallow, as well as what the various parameters should be until we
reached reasonable results.

    \hypertarget{abstract}{%
\subsection{Abstract}\label{abstract}}

We created a model to answer the question, ``What temperature and amount
of time will create the perfectly roasted marshmallow?'' and found
through a sweep of external temperature the following heat map of what
temperature and time roasted will create a marshmallow that is crispy
but not burnt and melted all the way through. To maximize time while
creating the perfect marshmallow, a marshmallow should be roasted for
380 seconds at 360 degrees Kelvin, or 188.33 degrees Fahrenheit.

    \begin{Verbatim}[commandchars=\\\{\}]
{\color{incolor}In [{\color{incolor}12}]:} \PY{n}{map\PYZus{}values}\PY{p}{(}\PY{n}{zone}\PY{p}{,} \PY{l+s+s1}{\PYZsq{}}\PY{l+s+s1}{Ideal Zone Heat Map}\PY{l+s+s1}{\PYZsq{}}\PY{p}{)}
\end{Verbatim}


    \begin{center}
    \adjustimage{max size={0.9\linewidth}{0.9\paperheight}}{output_20_0.png}
    \end{center}
    { \hspace*{\fill} \\}
    

    % Add a bibliography block to the postdoc
    
    
    
    \end{document}
